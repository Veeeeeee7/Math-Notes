\documentclass{article}
\usepackage{graphicx} % Required for inserting images
\usepackage{amsmath}
\usepackage{qtree}

\title{Multivariable Calculus Unit 4 Study Guide}
\author{}
\date{}
\setlength{\parindent}{0pt}

\begin{document}

\maketitle

\section{Double Integrals}
\subsection{Definition:}
\[
\iint\limits_{R} f(x,y)\, dA
\]
The area of the region $R$ under the surface formed by $f(x,y)$ is computed by summing the tiny areas $dA$.
\[
dA = dxdy \quad \text{or} \quad dA = dydx
\]

\subsection{Switching Order of Integration:}
\[
\int_{0}^{2} \int_{0}^{\sqrt{x}} f(x,y)\, dydx
\]
Given that $x$ ranges from 0 to 2, $y$ varies from 0 to $\sqrt{x}$. Thus the bound is $y=\sqrt{x} \to x=y^2$, $x=0$, and $x=2$. Plugging in $x=0$ and $x=2$ gives that $y$ ranges from 0 to $\sqrt{2}$. Given the bounds, $x$ varies from $y^2$ to 2.
\[
\int_{0}^{\sqrt{2}} \int_{y^2}^{2} f(x,y)\, dxdy
\]  

\section{Volume Given Bounds}
\subsection{Given integer bounds:}
\textbf{Example: }$\iint\limits_{R} xcos(xy)cos^2(\pi x)\, dA$, given $R = [0, \frac{1}{2}] \times [0, \pi ]$
\[
\int_{0}^{\frac{1}{2}} \int_{0}^{\pi} xcos(xy)cos^2(\pi x)\, dydx    
\]
\[
=\int_{0}^{\frac{1}{2}} \left[ cos^{2}(\pi x)sin(xy) \right]_{0}^{\pi} \, dx
\]
\[
=\int_{0}^{\frac{1}{2}} \left[ cos^{2}(\pi x)sin(\pi x) \right] \, dx
\]
\[
=\left[ -\frac{1}{3\pi} cos^{3}(\pi x) \right]_{0}^{\frac{1}{2}}
\]
\[
=\frac{1}{3\pi}
\]

\subsection{Given equation bounds for $z=f(x,y)$:}
\textbf{Example: }$x^2+y^2=9$, $z=0$, and $z=3-x$
\[
V=\iint\limits_{R} (3-x)\, dA
\]
Considering that $x$ ranges from $-3$ to $3$, then $y$ must range from $-\sqrt{9-x^2}$ to $\sqrt{9-x^2}$.
\[
V=\int_{-3}^{3} \int_{-\sqrt{9-x^2}}^{\sqrt{9-x^2}} (3-x)\, dydx
\]
\[
=\int_{-3}^{3} (6\sqrt{9-x^2}-2x\sqrt{9-x^2})\, dx
\]
\[
=27\pi
\]

\subsection{Switching to Polar:}
Switch $x$ to $r\cos(\theta)$ and $y$ to $r\sin(\theta)$. Switch $dA$ to $rdrd\theta$. Switch bounds from rectangular coordinates to polar coordinates.
\\
\\
\textbf{Example: }Find the volume of a sphere given the equation $x^2+y^2+z^2=9$ which is ouside the cylinder $x^2+y^2=1$.
\[
z=\sqrt{9-x^2-y^2}
\]
\[
V=\iint\limits_{R} \sqrt{9-x^2-y^2}\, dA = \iint\limits_{R} \left(\sqrt{9-r^2}\right) r\, drd\theta
\]
Given the bounds, $r$ ranges from 1 to 3, and $\theta$ ranges from 0 to $2\pi$.
\[
V=2*4\int_{0}^{\frac{\pi}{2}} \int_{1}^{3} \left(\sqrt{9-r^2}\right) r\, drd\theta
\]
\[
=8\int_{0}^{\frac{\pi}{2}} \frac{16\sqrt{2}}{3}\, d\theta
\]
\[
=\frac{64\sqrt{2}}{3}\pi
\]

\section{Area Given Bounds}
\subsection{Given equation bounds for $z=f(x,y)$:}
\textbf{Example: }Area under the surface $\frac{x}{\sqrt{1+y^2}}$ enclosed by $y=x^2$, $y=4$, and $x=0$.
\[
\iint\limits_{R} \frac{x}{\sqrt{1+y^2}}\, dA
\]
Considering that $y$ ranges from 0 to 4, then $x$ must range from 0 to $\sqrt{y}$.
\[
\int_{0}^{4} \int_{0}^{\sqrt{y}} \frac{x}{\sqrt{1+y^2}}\, dxdy
\]
\[
=\int_{0}^{4} \left[ \frac{x^2}{2\sqrt{1+y^2}} \right]_{0}^{\sqrt{y}}\, dy
\]
\[
=\int_{0}^{4} \frac{y}{2\sqrt{1+y^2}} \, dy
\]
\[
=\frac{\sqrt{17}-1}{2}
\]

\subsection{Given polar bounds:}
\textbf{Example: }The region enclosed by the cardioid $r=1-cos(\theta)$.
\[
\iint\limits_{R} 1\, dA = \iint\limits_{R} r\, drd\theta
\]
Given bounds, $\theta$ ranges from 0 to $2\pi$, and $r$ ranges from 0 to $1-cos(\theta)$.
\[
A=\int_{0}^{2\pi} \int_{0}^{1-cos(\theta)} r\, drd\theta
\]
\[
=\int_{0}^{2\pi} \frac{(1-cos(\theta))^2}{2}\, d\theta
\]
\[
=\frac{3\pi}{2}
\]

\section{Surface Area}
\textbf{Using Cross Product: }$dS=|\frac{\partial \vec{r}}{\partial u} \times \frac{\partial \vec{r}}{\partial v}|\,dudv$ \\
Given $f(u, v)$, find $\vec{r}(u, v)=\langle u, v, f(u, v) \rangle$.
\[
\frac{\partial \vec{r}}{\partial u}=\langle 1, 0, \frac{\partial f}{\partial u} \rangle
\]
\[
\frac{\partial \vec{r}}{\partial v}=\langle 0, 1, \frac{\partial f}{\partial v} \rangle
\]
\[
\left|\frac{\partial \vec{r}}{\partial u} \times \frac{\partial \vec{r}}{\partial v}\right|=\left|\left\langle \frac{\partial f}{\partial u}, \frac{\partial f}{\partial v}, 1 \right\rangle\right|=\sqrt{\left(\frac{\partial f}{\partial u}\right)^2+\left(\frac{\partial f}{\partial v}\right)^2+1}
\]

\textbf{Using Jacobian: }$dS=|J|\,dudv$, given $\vec{r}(u, v)=\langle \vec{r}_{1}(u, v), \vec{r}_{2}(u, v) \rangle$

\renewcommand{\arraystretch}{1.5}
\[
|J| = \left|\begin{array}{cc}
\frac{\partial \vec{r}_1}{\partial u} & \frac{\partial \vec{r}_2}{\partial u} \\
\frac{\partial \vec{r}_1}{\partial v} & \frac{\partial \vec{r}_2}{\partial v} \\
\end{array}\right|
\]

\subsection{Given integer bounds:}
\textbf{Example: }$y^2+z^2=9$, given $R=[0,2] \times [-3, 3]$
\[
z=\sqrt{9-y^2}
\]
\[
\vec{r}(x, y)=\langle x, y, \sqrt{9-y^2} \rangle
\]
\[
\frac{\partial \vec{r}}{\partial x}=\langle 1, 0, 0 \rangle
\]
\[
\frac{\partial \vec{r}}{\partial y}=\langle 0, 1, -\frac{y}{\sqrt{9-y^2}} \rangle
\]
\[
\left|\frac{\partial \vec{r}}{\partial x} \times \frac{\partial \vec{r}}{\partial y}\right|=\sqrt{1+\left(-\frac{y}{\sqrt{9-y^2}}\right)^2}=\sqrt{\frac{9-y^2+y^2}{9-y^2}}=\frac{3}{\sqrt{9-y^2}}
\]
\[
S=\int_{0}^{2} \int_{-3}^{3} \frac{3}{\sqrt{9-y^2}}\, dydx
\]
\[
=\int_{0}^{2} 3\pi\, dx
\]
\[
=6\pi
\]

\subsection{Given equation bounds:}
\textbf{Example: }$z^2=4x^2+4y^2$ bounded by $y=x$ and $y=x^2$
\[
\vec{r}(x, y)=\langle x, y, \sqrt{4x^2+4y^2} \rangle
\]
\[
2z\frac{\partial z}{\partial x}=8x \qquad 2z\frac{\partial z}{\partial y}=8y
\]
\[
\frac{\partial z}{\partial x}=\frac{4x}{z} \qquad \frac{\partial z}{\partial y}=\frac{4y}{z}
\]
\[
\frac{\partial \vec{r}}{\partial x}=\langle 1, 0, \frac{4x}{z} \rangle
\]
\[
\frac{\partial \vec{r}}{\partial y}=\langle 0, 1, \frac{4y}{z} \rangle
\]
\[
\left|\frac{\partial \vec{r}}{\partial x} \times \frac{\partial \vec{r}}{\partial y}\right|=\sqrt{\left(\frac{4x}{z}\right)^2+\left(\frac{4y}{z}\right)^2+1}=\sqrt{\frac{4(4x^2+4y^2)}{z^2}+1}=\sqrt{4+1}=\sqrt{5}
\]
Given the bounds, $x$ ranges from 0 to 1, and $y$ ranges from $x^2$ to $x$.
\[
S=\int_{0}^{1} \int_{x^2}^{x} \sqrt{5}\, dydx
\]
\[
=\int_{0}^{1} \sqrt{5}(x-x^2)\, dx
\]
\[
=\frac{\sqrt{5}}{6}
\]

\subsection{Switching to Polar:}
\textbf{Example: }$z=xy$ bounded by $y=\frac{x}{\sqrt{3}}$, $y=0$, and $x^2+y^2=9$
\[
\vec{r}(x,y)=\langle x, y, xy \rangle
\]
\[
\frac{\partial \vec{r}}{\partial x}=\langle 1, 0, y \rangle
\]
\[
\frac{\partial \vec{r}}{\partial y}=\langle 0, 1, x \rangle
\]
\[
\left|\frac{\partial \vec{r}}{\partial x} \times \frac{\partial \vec{r}}{\partial y}\right|=\sqrt{x^2+y^2+1}=\sqrt{r^2+1}
\]
Given the bounds, $r$ ranges from 0 to 3 because of the circle equation, and $\theta$ ranges from 0 to $\frac{\pi}{6}$ because of the line equation and that $tan^{-1}(\frac{1}{\sqrt{3}})=\frac{\pi}{6}$.
\[
S=\int_{0}^{\frac{\pi}{6}} \int_{0}^{3} (\sqrt{r^2+1})r\, drd\theta
\]
\[
=\int_{0}^{\frac{\pi}{6}} \frac{1}{3}(10\sqrt{10}-1)\, d\theta
\]
\[
=\frac{\pi}{18}(10\sqrt{10}-1)
\]

\end{document}
